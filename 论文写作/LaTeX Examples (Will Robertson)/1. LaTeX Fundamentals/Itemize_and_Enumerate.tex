% Copyright (C) Will Robertson 2004. 
%
% Permission is given to freely distribute and modify 
% this document provided attribution is given.

\documentclass[12pt,article]{memoir}

%\tightlists % this command will remove inter-item space from ALL lists 

\begin{document}

\chapter*{Basic lists}

\section*{Plain \LaTeX\ versions}

The \texttt{enumerate} environment:

\begin{enumerate}
	\item The first item
	\item The second item
	
	You can put paragraph text in between items
	
	\item The third item
\end{enumerate}

\noindent The \texttt{itemize} environment:

\begin{itemize}
	\item The first item
	\item The second item
	
	You can put paragraph text in between items
	
	\item The third item
\end{itemize}

\noindent The \texttt{description} environment.

\begin{description}
	\item[hello] a common greeting
	\item[goodbye] a common farewell
\end{description}

\section*{\textsf{Memoir} extensions}

The \textsf{memoir} class, as usual, has extended the functionality of the first two of the above lists. The functionality is based on the \textsf{enumerate} package. If you need even more options, you may look to the \textsf{paralist} package to solve your problems.

The new things you can do revolve mostly around list spacing and the symbols used before each item. To remove all inter-item space, put the \verb| \tightlists| command anywhere before your lists. (The \verb| \defaultlists| command will return things to normal.) Following are some of the things you can do with the new \texttt{enumerate} and \texttt{itemize} environments.

\subsection*{Some \texttt{enumerate} environments}
\tightlists
\begin{enumerate}[{A}1]
	\item This env. has an optional argument to change the look of it.
	\item The argument is \verb|[{A}1]|.
	\item The text in the curly brackets is included as is. (It can be anything.)
	\item The `1' is the index of the item. Other options are available\dots
\end{enumerate}

\begin{enumerate}[\scshape i]
	\item Index options are: \texttt{[1]} (default), \texttt{[i]},\texttt{[I]},\texttt{[A]}
	\item 1 goes 1,2,3,4,\dots
	\item i goes i,ii,iii,iv,\dots
	\item I goes I,II,III,IV,\dots
	\item A goes A,B,C,D,\dots
\end{enumerate}

\defaultlists

\subsection*{Some \texttt{itemize} environments}

\begin{itemize}
	\firmlist
	\item \textsf{Memoir} lets you adjust the spacing of your lists individually.
	\item This list contains a \verb|\firmlist| command, which reduces some of the inter-item spacing.
\end{itemize}

\begin{itemize}[\S]
	\item Option arguments for \texttt{itemize} allow you to customise the bullet for each item.
	\item You probably \emph{don't} want to use this list as an example.
\end{itemize}

\begin{itemize}[\$]
\tightlist
	\item This list contains a \verb|\tightlist| command, which removes all inter-item space.
	\item These spacing commands also work in \texttt{enumerate}.
	\item You can put anything you want for the bullet.
\end{itemize}



\end{document}