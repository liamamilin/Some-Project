% Copyright (C) Will Robertson 2004. 
%
% Permission is given to freely distribute and modify 
% this document provided attribution is given.

\documentclass[12pt,article]{memoir}

\begin{document}

\chapter*{Basic font selection}

This document is not about choosing different default fonts. It's simply about
what you can do with what you've got. You shouldn't have to use these commands
much. If you want to emphasise what you're saying, use the \verb|\emph{}|
command. That should be all you need in prose.

\section*{Font sizes}

It's important to use these commands to choose your fontsize so that if you ever
change the base font size of your document (with a class option), your larger
things will scale appropriately:

\begin{center}
  \tiny tiny
  \scriptsize scriptsize
  \footnotesize footnotesize
  \small small
  \normalsize normalsize
        
  \large large
  \Large Large
  \LARGE LARGE
  \huge huge
  \Huge Huge
\end{center}

\section*{Matching alternate fonts}

You may have noticed that I use a number of different fonts for things in this
document. They are all part of the same super-family ``Computer Modern'', so
they are designed to blend together nicely. See Table~\ref{tab:cm} for a list of
every font you may possibly choose with default \LaTeX\ font-changing commands.

\begin{table}[htbp]
\centering
\begin{tabular}{@{}ccc@{\hspace{2em}}>{\ttfamily}c>{\ttfamily}c@{}}
\toprule
\multicolumn{3}{c}{Font description} & \multicolumn{2}{c}{\LaTeX\ command}\\
\cmidrule(r){1-3}\cmidrule{4-5} Family & Series & Shape & \normalfont Closed &
\normalfont Open\\
\midrule Roman & Regular & Upright  & textrm & rmfamily \\
& & \itshape Italic & textit & itshape \\
& & \slshape Slanted & textsl & slshape  \\
& & \scshape Small Caps & textsc & scshape \\
& \bfseries Bold & \bfseries Upright & textbf & bfseries \\
& & \bfseries  \itshape Italic & \multicolumn{2}{c}{\ttfamily\emph{bf \& it}} \\
& & \bfseries  \slshape Slanted  & \multicolumn{2}{c}{\ttfamily\emph{bf \& sl}}
\\
\midrule
\sffamily San Serif & \sffamily Regular & \sffamily Upright & textsf & sffamily
\\
& & \sffamily \slshape Slanted  & \multicolumn{2}{c}{\ttfamily\emph{sf \& sl}}\\
& \sffamily \bfseries Bold & \sffamily \bfseries Upright 
&\multicolumn{2}{c}{\ttfamily\emph{sf \& bf}} \\
\midrule
\ttfamily Typewriter & \ttfamily Regular & \ttfamily Upright & texttt & ttfamily
\\
& & \ttfamily \itshape Italic  &\multicolumn{2}{c}{\ttfamily\emph{tt \& it}} \\
& & \ttfamily \slshape Slanted  & \multicolumn{2}{c}{\ttfamily\emph{tt \& sl}}
\\
& & \ttfamily \scshape Small Caps  & \multicolumn{2}{c}{\ttfamily\emph{tt \&
sc}} \\
\bottomrule
\end{tabular}
\caption{The Computer Modern font families. Italics are abbreviations meaning
combinations of commands given above them.}
\label{tab:cm}
\end{table}

There are two forms of font-choosing commands. The first is as I've used in the
source for Table~\ref{tab:cm}: \verb|{\slshape slanted}| to typeset {\slshape
slanted} (more often known as oblique); \verb|{\bfseries bold}| to typeset
{\bfseries bold}; etc.

The other method is for changing short amounts of text at a time, and the
commands are less complex than those above. \verb|\textsf{sans serif}| will
result in \textsf{sans serif}; \verb|\textsc{Small Caps}| will give
\textsc{Small Caps}; etc. \emph{This method is preferred,} because it
automatically takes into account `italic correction'. Italic correction is a
property of the font that corrects the spacing around a glyph that sticks out a
lot. You will rarely have to use it manually, but there are two examples shown
in Table~\ref{tab:itcorr}.

\begin{table}
  \centering
  \begin{tabular}{lc}
    \toprule
    Source & Output \\
    \midrule
    \verb|\textit{half}life| & \textit{half}life \\
    \verb|{\itshape half}life| & {\itshape half}life \\
    \verb|{\itshape half\/}life| & {\itshape half\/}life \\
    \midrule
    \verb|`\textbf{f}'| & `\textbf{f}' \\
    \verb|`{\bfseries f}'| & `{\bfseries f}' \\
    \verb|`{\bfseries f\/}'| & `{\bfseries f\/}' \\
    \bottomrule
  \end{tabular}
  \caption{Two examples demonstrating `italic correction'. Note that it's not only for italics!}
  \label{tab:itcorr}
\end{table}


\end{document}